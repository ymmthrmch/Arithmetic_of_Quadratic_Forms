\documentclass[a4paper,11pt]{jsarticle}

%%%%%%%%%%%%%%%%%%%%%%%input%%%%%%%%%%%%%%%%%%%%%%
\input{/Users/yamamotoharumichi/Library/TeXShop/Templates/preambles/environments.tex}
\input{/Users/yamamotoharumichi/Library/TeXShop/Templates/preambles/packages.tex}
\input{/Users/yamamotoharumichi/Library/TeXShop/Templates/preambles/theorem.tex}
\input{/Users/yamamotoharumichi/Library/TeXShop/Templates/preambles/operators_and_letters.tex}
\input{/Users/yamamotoharumichi/Library/TeXShop/Templates/preambles/layout_for_harumichi.tex}
\input{/Users/yamamotoharumichi/Library/TeXShop/Templates/preambles/spare.tex}
%%%%%%%%%%%%%%%%%%%%%%%input%%%%%%%%%%%%%%%%%%%%%%

%%%%%%%%%%%%%style of numbers in emumerate environment%%%%%%%%%%%%
\renewcommand{\labelenumi}{(\arabic{enumi})} %大番号の付け方
\renewcommand{\labelenumii}{(\roman{enumii})} %中番号の付け方
\renewcommand{\labelenumiii}{(\alph{enumiii})}%小番号の付け方
%%%%%%%%%%%%%style of numbers in emumerate environment%%%%%%%%%%%%

\begin{document}

%%%%%%%%%%%%%%%%%%%%%%%%title%%%%%%%%%%%%%%%%%%%%%%%%
\title{}
\author{山本晴道}
\date{最終更新:\today}
\maketitle
%%%%%%%%%%%%%%%%%%%%%%%%title%%%%%%%%%%%%%%%%%%%%%%%%

%%%%%%%%%%%%%%%%%%%%%content%%%%%%%%%%%%%%%%%%%%%
\setcounter{section}{17}
\section{単純・半単純代数}
\begin{set}
	\begin{itemize}
		\item $F$:体.
		\item $A$:$F$代数.
		\item $V,W$:(特に断りがなければ左)$A$加群.
	\end{itemize}
\end{set}
有限次元ベクトル空間,すなわち有限生成な体上の加群は
\begin{itemize}
	\item 1次元ベクトル空間という"素材"の和で表すことができる.
	\item 必要な"素材"は1次元ベクトル空間の1種類のみ.
\end{itemize}
などのいい性質を多く持つ.加群もこのような性質を持っていれば嬉しいが,もちろん一般には成り立たない.そこでその環上の加群がベクトル空間のような性質を持つという意味で体と類似している環のクラスをうまく定め,その性質を調べよう.それを定式化することができれば多少簡単に加群を扱うことができるだろう.まず環への適当な条件として,以下に定める$F$代数であることを課す.これにより加群をベクトル空間とみなすことができ,使える道具が増える.簡単に言えば$F$代数とは体$F$を中心に含む環のことである.
\begin{dfn}[$F$代数]
	\begin{enu}
		\item $\bbZ$加群$(A,+)$と演算$\cdot$,写像$\gvph:F\to \End A$の組であって以下を満たすものを$F$代数という.
		\begin{enu}
			\item $(A,+,\gvph)$は$F$ベクトル空間.
			\item $(A,+,\cdot)$は(単位的)環.
			\item $\forall \ga,\gb\in F,\forall x,y\in A, (\ga x)(\gb y)=(\ga\gb)(xy)$.
		\end{enu}
		\item 唯一の単射$F\into A$により$F\subseteq A$とみなす.
		\item $A$の元が全て可逆であるとき,$A$を可除代数という.
	\end{enu}
\end{dfn}
\begin{rmk}
	最初にいくつか注意をしておく.基本的に右作用や右加群に対して成り立つことは適切に書き換えることで左作用や左加群に対しても成り立つ.また$a$を$v$に右作用させるとき$va$と書くが明示したいときは$v\cdot a$と書く.
\end{rmk}
$F$代数$A$上の加群$W$の構造を調べていく.まずそのための道具として自己準同型代数を用意する.
\begin{dfn}[自己準同型代数]
	$W$を右$A$加群とする.
	\begin{itemize}
		\item $\End_A(W):=\{\gvph:W\to W\mid A\text{線型}\}$.
		\item $(\gvph +\gps)(x):=\gvph(x)+\gps(x)$.
		\item $(\gvph\gps)(x):=\gps(\gvph(x))$.
		\item $(\ga\gvph)(x):=\ga\gvph(x)$.
	\end{itemize}
	これらにより$\End_A(W)$は$F$代数であり,$W$に左から作用する.\\
	$W$が左$A$加群の場合,
	\[
		(\gvph\gps)(x):=\gps(\gvph(x)),\quad (\ga\gvph)(x):=\ga\gvph(x)
	\]
	と定め,$W$に右から作用する.
\end{dfn}
\begin{rmk}\label{rmk:end}
	\begin{enu}
		\item $W$と$A$の有限性は$\End_A(W)$に遺伝する.
		\begin{align*}
			W\text{が有限生成}A\text{加群かつ}\dim_F A<\infty & \iff\dim_F W<\infty               \\
			                                         & \iff \dim_F \End_F(W)<\infty      \\
			                                         & \implies \dim_F \End_A(W)<\infty.
		\end{align*}
		\item $A$を右$A$加群とみなしたとき,$\End_A(A)$は左スカラー倍全体であり,$A$と$F$同型である.具体的には
		\begin{align*}
			\End_A(A)\ni f\longmapsto f(1)\in A \\
			\End_A(A)\ni (x\mapsto ax)\longmapsfrom a \in A
		\end{align*}
		が$F$同型を与える.
		\item\label{item:end_of_end}
		$Y=\{x\in A\mid Vx=0\}$とおくと$Y$は$A$の両側イデアル.$A/Y$は作用が同じ元を同一視したもので,$V\rightaction A/Y$である.$A/Y$は$C$の部分代数であり,$B=D$である.
		\begin{align*}
			V                        & \rightaction A          &  & \text{:$A$の作用}         \\
			B=\End_A(V)\leftaction V &                         &  & \text{:$A$と可換な自己準同型全て} \\
			V                        & \rightaction\End_B(V)=C &  & \text{:$A$と可換な自己準同型全て} \\
			D=\End_C(V)\leftaction V &                         &  & \text{:$A$と可換な自己準同型全て}
		\end{align*}
	\end{enu}
\end{rmk}
\begin{prop}[有限直和の自己準同型代数]\label{prop:ds_and_end}
	$V$を左$A$加群,$W=\tp V^r$($V$成分の$r$次行ベクトル全体)とする.このとき$F$代数として
	\[
		\End_A(W)\cong \m_r(\End_A(V)).
	\]
	また証明で構成する同型によって誘導される右作用$W\rightaction\End_A(W)$は,通常の積$a\cdot w=aw$である.ただし$a_{ij}\in\End_A(V),v_j\in V$の積は$v_j\cdot a_{ij}$とする.
\end{prop}
\begin{proof}
	$i=1,\ldots,r$に対して$\gi_i:V\to W$を第$i$成分への埋め込み,$\gp_i:W\to V$を第$i$成分への射影とする.
	\[
		\gvph:\End_A(W)\ni \ga\to(\gp_j\circ\ga\circ\gi_i)_{i,j}\in\m_r(\End_A(V))
	\]
	と定める.これが同型を与えることを見る.まず$\gp_j\circ\ga\circ\gi_i\in\End_A(V)$であるので$\gvph$はwell-definedである.$\gvph$が和とスカラー倍を保存することもわかる.積についても
	\begin{align*}
		\gvph(\ga)\gvph(\gb) = & (\gp_j\circ\ga\circ\gi_i)_{i,j}(\gp_l\circ\gb\circ\gi_k)_{k,l}                                  \\
		=                      & \left(\sum_{j=1}^r(\gp_j\circ\ga\circ\gi_i)(\gp_l\circ\gb\circ\gi_j)\right)_{i,l}               \\
		=                      & \left(\sum_{j=1}^r(\gp_l\circ\gb\circ\gi_j)\circ(\gp_j\circ\ga\circ\gi_i)\right)_{i,l}          \\
		=                      & \left(\gp_l\circ\gb\circ\left(\sum_{j=1}^r\gi_j\circ\gp_j\right)\circ\ga\circ\gi_i\right)_{i,l} \\
		=                      & \left(\gp_l\circ\gb\circ\ga\circ\gi_i\right)_{i,l}                                              \\
		=                      & (\gp_l\circ\ga\gb\circ\gi_i)_{i,l}                                                              \\
		=                      & \gvph(\ga\gb).
	\end{align*}
	より保たれる.単射性は,$\ga,\gb\in\End_A(W)$について
	\begin{align*}
		\gvph(\ga) = \gvph(\gb) & \implies \exists i,j\in\{1,\ldots,r\}, \gp_j\circ\ga\circ\gi_i \neq \gp_j\circ\gb\circ\gi_i \\
		                        & \implies \exists i\in\{1,\ldots,r\},\exists v\in V, \ga(\gi_i(v)) \neq \gb(\gi_i(v))        \\
		                        & \implies \ga\neq \gb.
	\end{align*}
	よりしたがう.全射性は,任意の$(a_{ij})_{i,j}\in\m_r(\End_A(V))$について$\ga:W\to W$を
	\[
		\ga:(v_1,\ldots,v_r)\mapsto \left(\sum_{j=1}^r v_j\cdot a_{j1},\ldots,\sum_{j=1}^r v_j\cdot a_{jr}\right)
	\]
	と定めると$\ga\in\End_A(W)$であり,$\gvph(\ga)=(a_{ij})_{i,j}$となることよりしたがう.
\end{proof}
\begin{eg}[斜体の直和加群]\label{eg:ds_of_skew}
	ここまでの内容を確認する例を見る.\textbf{この例はあとで重要となる}.$D$を可除$F$代数とする.$V=\tp D^r$,$A=\m_r(D)$とおく.
	\begin{itemize}
		\item 左からの積で左作用$D\leftaction V$を定め$V$を左$D$加群とする.
		\item 右からの積で右作用$V\rightaction A$を定め$V$を右$A$加群とする.
	\end{itemize}
	$\End_D(V),\End_A(V)$を求めよう.$V$は(左からの積による)左$D$加群$D$の$r$個の直和ゆえ,命題\ref{prop:ds_and_end}より
	\[
		\End_D(V)\cong\m_r(\End_D(D))\cong\m_r(D).
	\]
	$\End_A(V)$について考える.$D\subseteq A$に制限すると命題\ref{prop:ds_and_end}より$\End_{D(\subseteq A)}(V)\cong\m_r(D)$である.$\End_{D(\subseteq A)}(V)\supseteq\End_A(V)$ゆえ,$\End_A(V)$は$\m_r(D)$のうち$A$の作用と可換なもの全体である.この条件を書き下していこう.$\End_{D(\subseteq A)}(V)=\m_r(D)$の左作用は命題\ref{prop:ds_and_end}を適切に左右逆に書き換えると,$a\in\m_r(D),v\in V$として$a\cdot v=v^{\op}a^{\op}$である.ただし$(-)^{\op}$は引数の各成分を反対環の元とみなす演算子である.$A$の右作用は$v\cdot a=va$であった.よって
	\[
		(b\cdot v)\cdot a = b\cdot (v\cdot a) \iff (v^{\op}b^{\op})^{\op}a=(va)^{\op}b^{\op}.
	\]
	$v$を$e_i=(0,\ldots,0,1,0,\ldots,0)\in V$とすると左辺は$ba$の第$i$行目,右辺は$a^{\op}b^{\op}$の第$i$行目である.したがって
	\[
		b\in \End_A(V)\implies \forall a\in A, ba=a^{\op}b^{\op}.
	\]
	$b\in \m_r(D)\setminus D$ならばこれを満たさないことがわかる.また注意\ref{rmk:end}(\ref{item:end_of_end})より$D\subseteq \End_A(V)$である.よって$\End_A(V)=D$.上記の内容をまとめておく.
	\begin{itembox}[l]{\textbf{例\ref{eg:ds_of_skew}.斜体の直和加群}}
		$D$を可除$F$代数,$V=\tp D^r,A=\m_r(D)$とする.標準的な積により$D\leftaction V\rightaction A$とする.このとき$\End_D(V)=A,\End_A(V)=D$である.
		\begin{align*}
			D\leftaction V           &                         &  & \text{:$D$の左作用}        \\
			V                        & \rightaction\End_D(V)=A &  & \text{:$D$と可換な自己準同型全て} \\
			\End_A(V)=D\leftaction V &                         &  & \text{:$A$と可換な自己準同型全て}
		\end{align*}
	\end{itembox}
\end{eg}
自己準同型代数の準備が終わったので,ベクトル空間の性質の抽出に取り掛かろう.まず1次元ベクトル空間の性質を抽出し"素材"に当たるものを定める.
\begin{dfn}[既約加群]
	$A$加群$V$が以下を満たすとき既約であるという.
	\begin{itemize}
		\item$V\neq 0$.
		\item 非自明な部分加群を持たない.
	\end{itemize}
\end{dfn}
\begin{eg}
	既約なベクトル空間は1次元ベクトル空間のみである.
\end{eg}
\begin{prop}[既約加群の性質]
	既約な$A$加群$V,W,W_i$に対して以下が成り立つ.
	\begin{enu}
		\item (既約ならば有限生成)$V$は$A$上有限生成.
		\item\label{prop:0_or_iso} (Schurの補題)$\forall \gvph\in \hom_A(V,W), \gvph = 0$または同型.
		\item (自己準同型環は可除環)$\End_A(V)$は可除代数.
	\end{enu}
\end{prop}
\begin{proof}
	\begin{enu}
		\item $x\in V$について$xA\leq A$ゆえ$x\neq 0$なら$xA=V$.
		\item $\ker \gvph\leq V,\im\gvph\leq W$である.$\ker \gvph = 0$かつ$\im \gvph = W$,または$\ker \gvph = V$かつ$\im \gvph = 0$であり,前者なら$\gvph$は同型,後者ならば$\gvph = 0$.
		\item (\ref{prop:0_or_iso})よりしたがう.
	\end{enu}
\end{proof}
\begin{eg}
	$A$が体$K$のとき,既約な$K$ベクトル空間は$K$のみ.$K$は$K$上1で生成されるため有限生成である.$K$準同型$K\to K$は1の像で決定される定数倍写像ゆえ同型か零写像である.$\hom_K(K,K)\cong K$ゆえ確かに可除である.
\end{eg}
ベクトル空間は"素材"である1次元ベクトル空間の和で表すことができたことを思い出し,続いて"素材"である既約加群の和で表せる空間を定義する.
\begin{dfn}[完全可約]
	$A$加群$V$が既約$A$加群の(有限とは限らない)和になるとき完全可約であるという.
\end{dfn}
\begin{rmk}
	0は0個の既約加群の和であるので完全可約である.
\end{rmk}
\begin{eg}
	任意のベクトル空間は(選択公理を採用すれば)基底を持つため完全可約である.
\end{eg}
\begin{prop}[完全可約加群の性質]\label{prop:complete_summand}
	\begin{enu}
		\item 以下は同値である.
		\begin{enu}
			\item $V$は完全可約である.つまり既約な部分加群の族$\{V_i\}_{i\in I}$があり
			\begin{equation}
				V=\sum_{i\in I}V_i\label{eq:complete_summand}
			\end{equation}
			となる.
			\item $V$では完全可約な補空間を取れる.つまり\[
				\forall U\leq V,\exists W\leq V, W\text{は完全可約かつ}V=U\oplus W.
			\]
			より詳しく(\ref{eq:complete_summand})のとき,$W=\sum_{i\in J}V_i\,(J\subseteq I)$とかける.
			\item  $V$は直和で完全可約である.つまり
			\[
				\exists \{W_i\}_{i\in J}, W_i\text{は既約かつ}V=\bigoplus_{i\in J}W_i.
			\]
			より詳しく(\ref{eq:complete_summand})のとき,$\{W_i\}_{i\in J}\subseteq \{V_i\}_{i\in I}$とできる.
		\end{enu}
		\item 完全可約性は和,部分,商に遺伝する.つまり完全可約加群$V,V_i$と$U\leq V$について,$\sum_{i\in I}V_i,U,V/U$は完全可約.
	\end{enu}
\end{prop}
\begin{proof}
	\begin{enu}
		\item
		\underline{(i)$\implies$(ii)} 既約な部分加群の族$\{V_i\}_{i\in I}$を用いて$V=\sum_{i \in I}V_i$と表せる.
		\[
			S=\left\{J\subseteq I\,\bigg|\, U\cap\sum_{i\in J}V_i=0\right\}
		\]
		にZornの補題を用いることで$S$の極大元$J$を得る.$X:= U\oplus  \sum_{j\in J} V_j$とおく.$X\neq V$と仮定すると,$V_k\not\subseteq X$なる$k\in I\setminus J$が存在する.すると$J\subsetneq J\cup\{k\}\in S$となる.なぜなら$0\leq V_k\cap X< V_k$と$V_k$が既約であることから$V_k\cap X=0$であり,
		\[
			v+w\in\left(V_k+\sum_{i\in J}V_i\right)\cap U\quad \left(v\in V_k, w\in \sum_{i\in J}V_i\right)
		\]
		ならば$v\in X\cap V_k$ゆえ$v=w=0$であり,したがって$(V_k+\sum_{i\in J}V_i)\cap U=0$であるから.しかしこれは$J$の極大性に矛盾.よって$X=V$.\\
		\underline{(ii)$\implies$(i)}\,$U=0$とすれば良い.\\
		\underline{(i)かつ(ii)$\implies$(iii)}\,(i)より既約な部分加群$\{V_i\}_{i\in I}$を用いて$V=\sum_{i \in I}V_i$と表せる.
		\[
			T=\left\{J\subseteq I \,\bigg|\,\sum_{i\in J}V_i=\bigoplus_{i\in J}V_i\right\}
		\]
		にZornの補題を用いることで$T$の極大元$J$を得る.$X:=\bigoplus_{j\in J}V_j$とおく.(ii)より$V=X\oplus \sum_{i\in K}V_i$となる$K\subsetneq I\setminus J$を取れる.$K\neq \varnothing$ならば$k\in K\setminus J$について$J\cup\{k\}\in T$となるがこれは$J$の極大性に矛盾.よって$K=\varnothing$であり$V=X$.
		\item
		和については明らか.(1)より$V= U\oplus U'$なる完全可約部分群$U'$が存在する.$V/U\cong U'$ゆえ$V/U$は完全可約.$U'$の完全可約な補空間を$U''$とおくと,$U\cong V/U'\cong U''$より$U$は完全可約.
	\end{enu}
\end{proof}
(1)の(ii)が完全可約の定義に採用されることもある.体の場合には,任意のベクトル空間が完全可約となった.そこでその環上の任意の加群が完全可約になるという意味で体に類似している代数のクラスを定めよう.
\begin{dfn}[半単純代数]
	任意の右(resp. 左)$A$加群が完全可約であるとき,$F$代数$A$を右(resp. 左)半単純であるという.
\end{dfn}
後ほどわかるが,実は右半単純であることと左半単純であることは同値になる.よって単に半単純ということもある.体の類似物を定めることができたのでその性質を見ていこう.
\begin{prop}[半単純代数の特徴付け]
	$F$代数$A$について
	\[
		A\text{が右半単純}\iff A\text{が右}A\text{加群として完全可約}
	\]
	である.
\end{prop}
\begin{proof}
	\underline{$\implies$} 定義よりしたがう.
	\underline{$\impliedby$} 任意に右$A$加群$V$をとる.
	\[
		\gvph:\bigoplus_{V}A\ni(a_v)_{v\in V}\mapsto \sum_{v\in V} a_vv\in V
	\]
	とおくと,$V\cong \bigoplus_{V}A/\ker \gvph$である.命題(\ref{prop:complete_summand})よりしたがう.
\end{proof}
ベクトル空間の場合,"素材"は1次元ベクトル空間の1種類のみだった.半単純代数上の加群の場合は実は1種類とは限らず,半単純代数のイデアルを用いて記述できる.
\begin{dfn}[極小イデアル]
	$F$代数$A$の右イデアルが右$A$加群として既約であるとき,極小右イデアルという.
\end{dfn}
\begin{prop}[半単純代数の既約加群と極小イデアル]
	$A$が右半単純であるとする.$A$は完全可約ゆえ極小右イデアルの族$\{V_i\}_{i\in I}$を用いて$A=\bigoplus_{i\in I} V_i$と書ける.このとき$A$の既約右$A$加群は$V_i$のいずれかと同型である.
\end{prop}
\begin{proof}
	既約右$A$加群$V$を任意にとる.$v\in V\setminus\{0\}$をとる.$vA=V$ゆえ$\gvph:A\ni x\mapsto vx\in V$は全射準同型.$J\subseteq I$があり,$A\cong \ker \gvph \oplus \bigoplus_{j\in J} W_j$となる.$\gvph$の全射性より$V\cong A/\ker\gvph \cong  \bigoplus_{j\in J} W_j$だが,$\#J \neq 1$なら$V$が既約であることに矛盾する.よって一意的に$j\in J$が存在し$V\cong W_j$.
\end{proof}
半単純代数は"素材"が1種類ではないという意味でまだ体とは類似し切ってはいない.そこで"素材"が1種類であるものを定義しよう.
\begin{dfn}[単純代数]
	$F$代数が右半単純であり,$0$と自身以外に右イデアルを持たないとき,単純であるという.
\end{dfn}
この条件は目的に対して強すぎるように見えるかもしれない(0,自身以外に右イデアルがあっても極小右イデアルの右加群としての同型類は1つかもしれない).しかし実はこのあと見るようにこれが目的のために必要十分な条件である.また先ほども述べたように右半単純性と左半単純性は同値ゆえ,単に"単純"という言葉遣いをしているが,それを見るまでは明示的に右単純と書くことにする.\\
単純代数は構造をある程度決定できる.そのためにいくつか準備をしよう.
\begin{lem}
	$D$を可除$F$代数とする.このとき
\end{lem}
% lemma18.9->thm18.11->simpleの定義->thm18.12
%%%%%%%%%%%%%%%%%%%%%content%%%%%%%%%%%%%%%%%%%%%
%\input{thebibliography.tex}
\end{document}






























































